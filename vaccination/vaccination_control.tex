\documentclass[english,12pt,letter]{article}

\usepackage{amsthm}
\usepackage[latin9]{inputenc}
\usepackage{babel}
\usepackage[hmargin=0.9in,vmargin=1.25in]{geometry}
\usepackage{graphicx}
\usepackage{subfigure}
\usepackage[colorlinks=true,citecolor=blue,urlcolor=blue]{hyperref}
\usepackage{amsmath}
\usepackage{amssymb,amsfonts}

\newtheorem{thm}{Theorem}
\newtheorem{lem}{Lemma}
\newtheorem{cor}{Corollary}
\newtheorem{dfn}{Definition}

\newcommand{\Rnot}{\sigma_0}
\newcommand{\xinf}{x^\infty}
\newcommand{\dom}{{\mathcal D}}
\newcommand{\R}{{\mathbb R}}
\newcommand{\xopt}{x_\text{opt}}
\newcommand{\yopt}{y_\text{opt}}
\newcommand{\ymax}{y_\text{max}}
\newcommand{\xoneinf}{x_1^\infty}
\newcommand{\xtwoinf}{x_1^\infty}
\newcommand{\xonetwoinf}{x_{1,2}^\infty}
\newcommand{\sdummy}{\hat{s}}

\DeclareMathOperator\supp{supp}

\begin{document}
\title{Optimal control of a multi-cohort SIR epidemic through scarce vaccination over time}
\author{
  David I. Ketcheson\thanks{Computer, Electrical, and Mathematical Sciences \& Engineering Division,
King Abdullah University of Science and Technology, 4700 KAUST, Thuwal
23955, Saudi Arabia. (david.ketcheson@kaust.edu.sa)}
}
\maketitle


\section{Problem description and assumptions}
In this work we consider an extension of the
classical SIR model of Kermack \& Mckendrick \cite{kermack1927contribution}, in which
the general population is divided into multiple groups, or cohorts, depending on
factors like risk or geographic location.  We assume that a vaccine becomes
available in limited quantities over time, and consider the problem of choosing
how best to distribute the vaccine among the various cohorts.

The population is divided into susceptible ($x$), infected ($y$), and removed ($z$) fractions.
We assume the epidemic occurs on a timescale over which the the overall population
and the individual cohorts do not change their composition, so that
\begin{align}
    \sum_i (x_i(t) + y_i(t) + z_i(t))dt = 1.
\end{align}
The model takes the form
\begin{subequations} \label{SIR}
\begin{align} 
    x_i'(t) & = -x_i(t) \sum_j \beta_{ij}  y_j(t) - u_i(t) \\
    y_i'(t) & = x_i(t) \sum_j \beta_{ij} y_j(t) - \gamma y_i(t) \label{eq:y} \\
    z_i'(t) & = \gamma y_i(t) + u_i(t) \label{eq:z}.
\end{align}
\end{subequations}
Here $u_i(t)$ are the control variables which indicate the rate of vaccination
among cohort $i$.  The values $\beta_{ij}$ are the statistical contact rates
between the different cohorts.
The total amount of vaccine available up to time $t$ is denoted
by $a(t)$.  Vaccination is constrained by
\begin{align}
    \sum_i \int_0^t u_i(t) dt \le a(t).
\end{align}

Our first objective function is simply to minimize the number of deaths.  Let
$r_i$ denote the infection fatality ratio for cohort $i$.
Then the number of deaths in cohort $i$ up to time $T$ is
\begin{align}
 &  r_i \left(z_i(T) - \int_0^T u_i(t) dt\right) \\
 = & r_i\left(p_i-x_i - \int_0^T u_i(t) dt\right),
\end{align}
where $p_i$ is the total population of cohort $i$.
We thus take as objective function (omitting the constant term $r_i p_i$)
\begin{align}
    J & = - \sum_i r_i \left( x_i(T) + \int_0^T u_i(t) dt\right).
\end{align}
The Hamiltonian is then
\begin{align}
    H = - \sum_i \lambda_i x_i \sum_j \beta_{ij} y_j + \sum_i \mu_i x_i \sum_j \beta_{ij} y_j 
            - \sum_i \mu_i \gamma y_i - \sum_i (\lambda_i + r_i) u_i.
\end{align}
Here $\lambda_i, \mu_i$ are the adjoint variables, which satisfy
\begin{align}
    \lambda_i'(t) & = -\frac{\partial H}{\partial x_i} = (\lambda_i - \mu_i) \sum_j \beta_{ij} y_j \\
    \mu_i'(t) & = -\frac{\partial H}{\partial y_i} = \sum_j (\lambda_j - \mu_j) \beta_{ji} x_j + \gamma \mu_i \\
    \lambda_i(T) & = - r_i \\
    \mu_i(T) & = 0.
\end{align}


\section{Extended compartmental model}
Now we consider a model including additional compartments: exposed,
recently vaccinated, symptomatic, asymptomatic, known recovered, and unknown recovered.
When the vaccine administered, it is distributed randomly among the susceptible,
asymptomatic, and unknown recovered groups.  Contact rates for the symptomatic
group are significantly lowered.


\bibliographystyle{plain}
\bibliography{refs}

\end{document}
