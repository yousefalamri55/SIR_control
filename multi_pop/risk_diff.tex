\documentclass[english,12pt,letter]{article}

\usepackage{amsthm}
\usepackage[latin9]{inputenc}
\usepackage{babel}
\usepackage[hmargin=0.9in,vmargin=1.25in]{geometry}
\usepackage{graphicx}
\usepackage{subfigure}
\usepackage[colorlinks=true,citecolor=blue,urlcolor=blue]{hyperref}
\usepackage{amsmath}
\usepackage{amssymb,amsfonts}

\newtheorem{thm}{Theorem}
\newtheorem{lem}{Lemma}
\newtheorem{cor}{Corollary}
\newtheorem{dfn}{Definition}

\newcommand{\Rnot}{\sigma_0}
\newcommand{\xinf}{x^\infty}
\newcommand{\dom}{{\mathcal D}}
\newcommand{\R}{{\mathbb R}}
\newcommand{\xopt}{x_\text{opt}}
\newcommand{\yopt}{y_\text{opt}}
\newcommand{\ymax}{y_\text{max}}
\newcommand{\xoneinf}{x_1^\infty}
\newcommand{\xtwoinf}{x_1^\infty}
\newcommand{\xonetwoinf}{x_{1,2}^\infty}
\newcommand{\sdummy}{\hat{s}}

\DeclareMathOperator\supp{supp}

\begin{document}
\title{Optimal control of an SIR epidemic through contact reduction with differing risk levels}
\author{
  David I. Ketcheson\thanks{Computer, Electrical, and Mathematical Sciences \& Engineering Division,
King Abdullah University of Science and Technology, 4700 KAUST, Thuwal
23955, Saudi Arabia. (david.ketcheson@kaust.edu.sa)}
}
\maketitle

\abstract{Motivated by the COVID-19 pandemic and responses to it,
we consider the problem of controlling an SIR-model epidemic
by temporarily reducing the rate of contact within a population.
We assume that the population is divided into groups with
different risks (e.g., different fatality ratios).
The control takes the form of a multiplicative reduction in the contact rate
of infectious individuals.  The control is allowed to be applied only over
a finite time interval, while the objective is to minimize the total impact
of the epidemic, given by the product of the number of infected individuals
from each group with the relative risk level of that group.
}

\section{Problem description and assumptions}
In this work we consider an extension of the
classical SIR model of Kermack \& Mckendrick \cite{kermack1927contribution}, in which
the population is parameterized by risk.  The risk parameter is denoted by $s\in\Lambda \subseteq R^d$.
and represents, e.g. the probability that an infection would lead to death.  This probability is
often known to a significant extent in terms of comorbidities or other risk factors such as age.
The population is divided into susceptible ($x$), infected ($y$), and removed ($z$) fractions.
We assume the epidemic occurs on a timescale over which the risk parametrization does
not change and the total population is constant, so that
\begin{align}
    x(s,t) + y(s,t) + z(s,t) = f(s); \ \ \ \int_\Lambda f(s) ds = 1.
\end{align}
The model takes the form
\begin{subequations} \label{SIR}
\begin{align} 
    x'(s,t) & = -\beta x(s,t) y(t) \\
    y'(s,t) & = \beta x(s,t) y(t) - \gamma y(s,t) \label{eq:y} \\
    (x(s,0),y(s,0)) & \in \dom := \{(x_0(s),y_0(s)) : x_0(s) \ge 0, y_0(s)\ge 0, \int_\Lambda (x_0(s)+y_0(s))ds \le 1\},
\end{align}
\end{subequations}
where $y(t)=\int_\Lambda y(s,t)ds$.  Let $\Rnot = \beta/\gamma$ denote the
basic reproduction number; we assume that $\Rnot>1$, which means that an epidemic of some
size will occur if an infected individual is introduced into a fully-susceptible population.

We introduce additionally a control parameter $q(s,\sdummy, t)$ corresponding to a reduction in contact
between susceptible individuals with risk $s$ and infected individuals with risk $\sdummy$.  This may represent population-wide non-pharmaceutical
interventions (NPIs) such as social distancing measures, as well as targeted measures such as
contact tracing and quarantining for contacts of known infected individuals.  The model then
takes the form
\begin{subequations} \label{SIRqc}
\begin{align} 
    x'(s,t) & = -\beta x(s,t) \int_\Lambda (1-q(s,\sdummy,t))y(\sdummy,t) d\sdummy \label{eq:x} \\
    y'(s,t) & = \beta x(s,t) \int_\Lambda(1-q(s,\sdummy,t))y(\sdummy,t)d\sdummy - \gamma y(s,t) \label{eq:y} \\
    (x(s,0),y(s,0)) & \in \dom,  \ \ \ q(s,\sdummy,t) \in [0,1].
\end{align}
\end{subequations}
It can be shown, by means similar to those used for the original SIR model, that
the region $\dom$ is forward-invariant and a unique solution exists for all
time (see e.g. \cite{hethcote2000mathematics}).

We wish to choose the control $q(s,t)$ so as to minimize the cost functional
\begin{align} \label{J-inf-time}
    J(x,y,q) := -\int_\Lambda c(s) \lim_{t\to\infty}x(s,t) ds + \int_0^\infty L(x,y,q) dt.
\end{align}
Here $c(s)$ represents the cost (e.g., the probability of death) associated
with infection of individuals with risk level $s$.

Since it is not reasonable to expect that intervention measures will be in place
indefinitely, we assume there is some finite time $T$ after which intervention stops:
\begin{align} \label{q-finite-time}
    q(s,\sdummy,t) & = 0 \ \ \ t>T.
\end{align}
This allows us to reformulate the objective in terms of the finite time $T$.  We define
\begin{align}
    \xinf(s,x(\cdot,T),y(\cdot,T),\Rnot) := \lim_{t\to\infty} \hat{x}(s,t)
\end{align}
where $\hat{x}(s,t)$ is the solution of \eqref{SIR} with initial data $x(\cdot,T), y(\cdot,T)$.
We further assume that $L=0$ for $t>T$; this assumption will be discussed later.
Then under the condition \eqref{q-finite-time}, the objective \eqref{J-inf-time} is
equivalent to
\begin{align} \label{J-infdim}
    J(x,y,q) := -\int_\Lambda c(\sdummy) \xinf(\sdummy,x(s,T),y(s,T),\Rnot) d\sdummy + \int_0^T L(x,y,q) dt.
\end{align}

We will focus on a simplification of the above model, based on the following considerations.
First, we suppose that targeted measures for quarantining infected or suspected individuals
are implemented uniformly across risk groups, so that $q$ depends only on $s$ and not $\sdummy$.
In practice the risk function is not known precisely, and NPIs cannot realistically be
implemented in a continuously-varying manner.  We thus take $\Lambda=[0,1]$ and
assume $c(s)$ to be a piecewise constant function:
\begin{align}
    c(s) & = c_j \text{ for } s_j < s < s_j+1,
\end{align}
where $0=s_1 < s_2 < \cdots < s_m=1$.
Then \eqref{SIRqc} reduces to a 2$m$-dimensional system of ODEs:
\begin{subequations} \label{SIRqd}
\begin{align} 
    x_j'(t) & = -\beta (1-q_j(t)) x_j(t)  y(t)  & j = 1, 2, \dots, m \\
    y_j'(t) & = \beta (1-q_j(t)) x_j(t)  y(t) - \gamma y_j(t) & j = 1, 2, \dots, m  \\
    & x_j(0)\ge 0,y_j(0) \ge 0, \sum_{j=1}^m (x_j(0)+y_j(0) \le 1 \ \ \ q_j(t) \in [0,1].
\end{align}
\end{subequations}
Given the inherent difficulty of imposing NPIs, and the increased difficulty related
to finely nuanced NPIs, the simplest case of just two risk groups (i.e. $m=2$)
is of particular interest and will be our main focus.  In this case \eqref{SIRqd} is
simply
\begin{subequations} \label{SIRq2}
\begin{align} 
    x_1'(t) & = -(1-q_1(t))\beta (y_1(t)+y_2(t)) x_1(t) \\
    x_2'(t) & = -(1-q_2(t))\beta (y_1(t)+y_2(t)) x_2(t) \\
    y_1'(t) & = (1-q_1(t))\beta (y_1(t)+y_2(t)) x - \gamma y_1(t) \\
    y_2'(t) & = (1-q_2(t))\beta (y_1(t)+y_2(t)) x - \gamma y_2(t) \\
    (x_1(0),x_2(0), y_1(0), y_2(0)) & \in \dom := \{(x_{10},x_{20},y_{10},y_{20}) : x_{10},x_{20},y_{10},y_{20} \ge 0, x_{10}+x_{20}+y_{10}+y_{20} \le 1\},
\end{align}
\end{subequations}
We assume that the infection poses strongly differing risks to the two groups;
for concreteness we will refer to population 1 as low-risk and population 2
as high risk.  
%Here risk means, for instance, the likelihood that an infection
%will result in fatality.  For instance, in the case of COVID-19, age is a
%strong predictor of risk.
The assumption \eqref{q-finite-time} becomes
\begin{align} \label{q-shortterm}
    q_{1,2}(t)=0 \text{ for } t>T,
\end{align}
and the objective \eqref{J-infdim} becomes
\begin{align} \label{eq:obj}
    J = -c_1 \xoneinf(T) - c_2 \xtwoinf(T) + \int_0^T L(y_{1,2},q)dt.
\end{align}
where
\begin{align}
    x_{1,2}^\infty = \lim_{t\to\infty} x_{1,2}(t).
\end{align}
It is convenient to define $x=x_1+x_2$ and $y=y_1+y_2$.
The system \eqref{SIRq2}, like system \eqref{SIR}, is at equilibrium if
$y(t)=0$.  Due to \eqref{q-shortterm}, we are interested in the stability
of this equilibrium when $q_1=q_2=0$.
In this case, the equilibrium is stable if and
only if $x(t)\le 1/\Rnot$, a condition referred to as {\em herd immunity}.
Since the system will eventually reach herd immunity, it is advantageous
to minimize the so-called {\em epidemiological overshoot}, which referes
to the excess population infected beyond the inevitable fraction $(\Rnot-1)/\Rnot$;
i.e. the overshoot is $\Rnot-\xinf$.
This was the main object of the previous study \cite{ketcheson2020}.
Herein, we have additionally the complication of the differing risks for the
two groups; in this setting it is more important to avoid infection of
the high risk group.

We state the control problem as follows:
\begin{align} \label{eq:basic-problem}
\begin{aligned}
& \text{Given } x_0, y_0, \sigma_0, T, \text{ choose } q(t) \in [0,1] \text{ to minimize \eqref{eq:obj}}  \\
& \text{ subject to } \eqref{SIRq2}.
\end{aligned}
\end{align}

The modeling and assumptions in the present work are motivated by the
current COVID-19 epidemic, which so far is being managed through broad
NPIs and without a vaccine.  In order to understand the effects of NPIs
imposed on an entire population, we stick to the simple model \eqref{SIRq2}
rather than explicitly modeling quarantined individuals.
Since such population-wide measures cannot be maintained indefinitely, we
invoke the finite-time control assumption \eqref{q-shortterm}.
This assumption is not new (see e.g. \cite{greenhalgh1988some}),
but unlike previous works our objective function is still based on the
long-term outcome (rather than the outcome at time $T$).
This drastically changes the nature of optimal solutions.
In the last section of this work we also consider the goal of avoiding
hospital overflow (through {\em flattening the curve}).

The rest of the paper is organized as follows.  In Section \ref{sec:prelims} we
review results on the exact solution of the SIR model and compute some quantities
that will be central to our results.  In Section \ref{sec:analytic} we state
and prove solutions of \eqref{eq:basic-problem} under certain choices of $J$.
In Section X we illustrate results for other choices of $J$ with computational
examples.

\section{Preliminaries\label{sec:prelims}}

\subsection{Notation}
The control above is given as $q_{1,2}(t)$, but for most purposes it is more
convenient to work with the effective reproduction number, defined as
$$
    \sigma_{1,2}(t) := (1-q_{1,2}(t))\Rnot = (1-q_{1,2}(t))\beta/\gamma.
$$
We will often simply refer to $\sigma_{1,2}(t)$ as the control.

In general, the solution of \eqref{SIRq2} depends on the initial data
$(x_0,y_0)$, the control $\sigma_{1,2}(t)$, and time $t$, so it is natural to write
$x(t;\sigma_{1,2}(t),x_0,y_0)$.
In what follows it will be convenient to make a slight abuse of notation and
write $x(t;\sigma_{1,2}(t))$ or $x(t)$ when there is no chance of confusion.

For a fixed reproduction number, the asymptotic susceptible fractions $\xoneinf, \xtwoinf$
can be obtained from the solution $v(t)$ at any time $t$.
Thus we will write $\xinf(x,y)$ or $\xinf(x,y,\Rnot)$.

\subsection{Formula for $x_{1,2}^\infty$ and derivatives}
To solve the problem described above, we must be able to compute
$\xonetwoinf$ from the terminal state $v(T)$.  Since $q(t)=0$ for $t>T$,
the values of $x=x_1+x_2$ and $y=y_1+y_2$ evolve according to \eqref{SIR}
after that time.  We have
$$
\frac{d x_1}{d x_2} = \frac{x_1}{x_2},
$$
which implies that $x_1/x_2$ is constant.  Thus
$$
\frac{\xoneinf}{\xtwoinf} = \frac{x_1(T)}{x_2(T)}.
$$
Furthermore, we have
$$
\xoneinf + \xtwoinf = \xinf(v(T),\Rnot).
$$
Combining these relations we find that
$$
    \xtwoinf = \xinf \frac{x_2(T)}{x(T)}.
$$
Thus we can write (setting $c_3=0$ for the moment):
\begin{align}
    J & = -c_1 \xoneinf - c_2 \xtwoinf = - \xinf \frac{c_1 x_1(T) + c_2 x_2(T)}{x(T)}.
\end{align}
We then find
\begin{align}
    \frac{\partial J}{\partial x_1} & = - \frac{\partial \xinf}{\partial x}\frac{c_1 x_1(T) + c_2 x_2(T)}{x(T)} 
            + \xinf x_2(T) \frac{c_2-c_1}{x(T)^2}, \label{eq:djdx1} \\
    \frac{\partial J}{\partial x_2} & = - \frac{\partial \xinf}{\partial x}\frac{c_1 x_1(T) + c_2 x_2(T)}{x(T)} 
            - \xinf x_1(T) \frac{c_2-c_1}{x(T)^2}. \label{eq:djdx2} \\
    \frac{\partial J}{\partial y_1} = \frac{\partial J}{\partial y_2} & = - \frac{\partial \xinf}{\partial y}\frac{c_1 x_1(T) + c_2 x_2(T)}{x(T)}.
\end{align}
An interesting observation can be made already, based on these derivatives.
The first term in both of these derivatives is negative whenever $x>1/\Rnot$.
Since $c_2>c_1$, the last term in \eqref{eq:djdx2} is negative.  This means that
increasing $x_2$ (the high-risk susceptible population) always improves the overall outcome.
But the last term in \eqref{eq:djdx1} is positive, so in some cases it may be that
increasing the low-risk susceptible population is detrimental to the overall outcome.


Let us define
$$
   \mu(x,y,\Rnot) := x(t) e^{-\Rnot(x(t)+y(t))},
$$
which is constant in time for any solution of \eqref{SIR}.
Since $y_\infty=0$, we have
$$
    x_\infty = x_0 e^{\Rnot(x_\infty-x_0-y_0)} = \mu(x_0,y_0,\Rnot) e^{\Rnot x_\infty}.
$$
Setting $w=-x_\infty \Rnot$ we have
$$
    we^w = -x_0 \Rnot e^{-\Rnot(x_0+y_0)} = -\mu \Rnot.
$$
Thus $w = W_0(-\mu\Rnot)$ where $W_0$ is the principal branch of Lambert's $W$-function \cite{pakes2015lambert},
and
\begin{align} \label{eq:xinf}
    x_\infty = -\frac{1}{\Rnot}W_0(-\mu \Rnot).
\end{align}
For the derivatives of $J$ above, we need the derivatives of $\xinf$ with respect to $x$ and $y$.
Direct computation gives
\begin{subequations} \label{xinf-grad}
\begin{align}
    \frac{\partial \xinf}{\partial y(t)} & = -\frac{\Rnot \xinf}{1-\Rnot \xinf} \\
    \frac{\partial \xinf}{\partial x(t)} & = \left(1-\frac{1}{x(t)\Rnot}\right) \frac{\partial \xinf}{\partial y(t)}
      = \frac{1-\Rnot x(t)}{1-\Rnot \xinf} \cdot \frac{\xinf}{x(t)}.
\end{align}
\end{subequations}


\bibliographystyle{plain}
\bibliography{refs}

\end{document}
